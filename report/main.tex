\documentclass[11pt,a4paper]{article}
\usepackage[margin=1in]{geometry}
\usepackage{graphicx}
\usepackage{hyperref}
\usepackage{float}
\usepackage{xcolor}
\usepackage{listings}

\lstset{
  basicstyle=\ttfamily\small,
  breaklines=true,
  frame=single,
  backgroundcolor=\color{gray!10},
  keywordstyle=\color{blue!60!black},
  commentstyle=\color{green!50!black}
}

\graphicspath{{images/}}

\title{Blue-Green Deployment of a Node.js Application}
\author{Prepared by Codex Assistant}
\date{\today}

\begin{document}
\maketitle

\section*{Objective}
Implement a zero-downtime blue-green deployment workflow for a Node.js service by containerizing the app, publishing images to Docker Hub, and automating deployment via a Jenkins pipeline that alternates between blue and green environments.

\section{System Overview}
\begin{itemize}
  \item \textbf{Application}: Express server exposing `/` and `/health`.
  \item \textbf{Containerization}: Dockerfile based on `node:20-alpine` plus `.dockerignore`.
  \item \textbf{Orchestration}: `docker-compose.bluegreen.yml` launches two app containers and an NGINX proxy that reads the active color from `.env`.
  \item \textbf{CI/CD}: Jenkins pipeline builds, tests, publishes, deploys to the idle color, smoke-tests, and optionally flips traffic.
\end{itemize}

\section{Implementation Steps}
\subsection{Build the Node.js service}
\begin{lstlisting}[language=bash]
npm init -y
npm install express
cat > src/server.js <<'JS'
  // Express app uses APP_COLOR/APP_VERSION to render status
JS
npm start
\end{lstlisting}

\subsection{Containerize the service}
\begin{lstlisting}[language=bash]
docker build -t bluegreen-demo:dev .
docker run -p 3000:3000 bluegreen-demo:dev
\end{lstlisting}

\subsection{Compose blue/green stacks}
\begin{lstlisting}[language=bash]
cp .env.sample .env  # fill Docker Hub values
ACTIVE_COLOR=blue docker compose -f docker-compose.bluegreen.yml up -d --build
open http://localhost:8080
\end{lstlisting}

\subsection{Automate release helpers}
\begin{itemize}
  \item \texttt{scripts/build-and-push.sh <tag>} builds/pushes \texttt{DOCKERHUB\_USERNAME/APP\_NAME:<tag>}.
  \item \texttt{scripts/deploy-color.sh <blue|green> <image> <version>} updates \texttt{.env} entries and restarts the chosen color.
  \item \texttt{scripts/switch-color.sh <blue|green>} rewrites \texttt{ACTIVE\_COLOR} and recreates the proxy.
\end{itemize}

\section{Jenkins Pipeline}
The declarative pipeline (Listing~\ref{lst:jenkins}) contains parameters for repository, tag, smoke-test URL, deploy color, and whether to switch traffic. Pipeline stages:
\begin{enumerate}
  \item Checkout code and install dependencies with `npm ci`.
  \item Run placeholder unit tests (ready for future expansion).
  \item Build Docker image tagged with either the provided \texttt{IMAGE\_TAG} or \texttt{build-\$BUILD\_NUMBER}.
  \item Push to Docker Hub using the \texttt{dockerhub-creds} credential.
  \item Deploy the idle color via `scripts/deploy-color.sh`.
  \item Smoke-test the `/health` endpoint.
  \item Optionally switch the proxy to route users to the freshly deployed color.
\end{enumerate}

\begin{lstlisting}[language=Java, caption={Jenkinsfile excerpt}, label={lst:jenkins}]
stage('Deploy to idle color') {
  steps {
    sh "./scripts/deploy-color.sh ${params.DEPLOY_COLOR} $IMAGE_NAME $RELEASE_TAG"
  }
}
\end{lstlisting}

\section{Testing and Verification}
\subsection{Docker build}
\begin{figure}[H]
  \centering
  \includegraphics[width=0.85\linewidth]{docker-build.png}
  \caption{Placeholder --- replace with actual Docker build/push screenshot.}
\end{figure}

\subsection{Blue vs Green containers}
\begin{figure}[H]
  \centering
  \includegraphics[width=0.85\linewidth]{compose-green.png}
  \caption{Placeholder --- capture `docker compose ps` showing both environments.}
\end{figure}

\subsection{Jenkins success}
\begin{figure}[H]
  \centering
  \includegraphics[width=0.85\linewidth]{jenkins-pipeline.png}
  \caption{Placeholder --- screenshot of Jenkins stage view/pipeline success.}
\end{figure}

\section{Results}
The lab delivers a reusable workflow:
\begin{itemize}
  \item Deterministic Node.js container image published to Docker Hub.
  \item Docker Compose stack that keeps blue and green environments warm while an NGINX proxy steers production traffic.
  \item Jenkins pipeline that automates build, push, deploy, smoke test, and promotion with a single click.
\end{itemize}

\section{Next Steps}
\begin{itemize}
  \item Replace placeholder screenshots with real captures before submission.
  \item Add automated tests (e.g., supertest) and integrate into the `npm test` stage.
  \item Extend pipeline with automated rollback logic if smoke tests fail.
\end{itemize}

\end{document}
