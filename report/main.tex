\documentclass[11pt,a4paper]{article}
\usepackage[margin=1in]{geometry}
\usepackage{graphicx}
\usepackage{float}
\usepackage{hyperref}
\usepackage{xcolor}
\usepackage{listings}
\usepackage{fancyhdr}

\lstset{
  basicstyle=\ttfamily\small,
  breaklines=true,
  frame=single,
  backgroundcolor=\color{gray!10},
  keywordstyle=\color{blue!60!black},
  commentstyle=\color{green!50!black}
}

\graphicspath{{images/}}

\title{Blue-Green Deployment of a Node.js Application}
\author{Yuvan Raj Krishna (Reg No. 22011102127)}
\date{\today}

\pagestyle{fancy}
\fancyhf{}
\fancyhead[L]{Yuvan Raj Krishna --- Reg No. 22011102127}
\setlength{\headheight}{14pt}
\cfoot{\thepage}

\begin{document}
\maketitle

\section*{Objective}
Implement a zero-downtime blue-green deployment workflow for a Node.js service by containerizing the app, publishing images to Docker Hub, and automating deployment via a Jenkins pipeline that alternates between blue and green environments.

\section{System Overview}
\begin{itemize}
  \item \textbf{Application}: Express server exposing `/` and `/health`.
  \item \textbf{Containerization}: Dockerfile based on `node:20-alpine` plus `.dockerignore`.
  \item \textbf{Orchestration}: `docker-compose.bluegreen.yml` launches two app containers and an NGINX proxy that reads the active color from `.env`.
  \item \textbf{CI/CD}: Jenkins pipeline builds, tests, publishes, deploys to the idle color, smoke-tests, and optionally flips traffic.
\end{itemize}

\section{Implementation Steps}
\subsection{Build the Node.js service}
\begin{lstlisting}[language=bash]
npm init -y
npm install express
cat > src/server.js <<'JS'
  // Express app uses APP_COLOR/APP_VERSION to render status
JS
npm start
\end{lstlisting}

\subsection{Containerize the service}
\begin{lstlisting}[language=bash]
docker build -t bluegreen-demo:dev .
docker run -p 3000:3000 bluegreen-demo:dev
\end{lstlisting}

\subsection{Compose blue/green stacks}
\begin{lstlisting}[language=bash]
cp .env.sample .env  # fill Docker Hub values
ACTIVE_COLOR=blue docker compose -f docker-compose.bluegreen.yml up -d --build
open http://localhost:8080
\end{lstlisting}

\subsection{Automate release helpers}
\begin{itemize}
  \item \texttt{scripts/build-and-push.sh <tag>} builds/pushes \texttt{DOCKERHUB\_USERNAME/APP\_NAME:<tag>}.
  \item \texttt{scripts/deploy-color.sh <blue|green> <image> <version>} updates \texttt{.env} entries and restarts the chosen color.
  \item \texttt{scripts/switch-color.sh <blue|green>} rewrites \texttt{ACTIVE\_COLOR} and recreates the proxy.
\end{itemize}

\section{Jenkins Pipeline}
The declarative pipeline (Listing~\ref{lst:jenkins}) contains parameters for repository, tag, smoke-test URL, deploy color, and whether to switch traffic. Pipeline stages:
\begin{enumerate}
  \item Checkout code and install dependencies with `npm ci`.
  \item Run placeholder unit tests (ready for future expansion).
  \item Build Docker image tagged with either the provided \texttt{IMAGE\_TAG} or \texttt{build-\$BUILD\_NUMBER}.
  \item Push to Docker Hub using the \texttt{dockerhub-creds} credential.
  \item Deploy the idle color via `scripts/deploy-color.sh`.
  \item Smoke-test the `/health` endpoint.
  \item Optionally switch the proxy to route users to the freshly deployed color.
\end{enumerate}

\begin{lstlisting}[language=Java, caption={Jenkinsfile excerpt}, label={lst:jenkins}]
stage('Deploy to idle color') {
  steps {
    sh "./scripts/deploy-color.sh ${params.DEPLOY_COLOR} $IMAGE_NAME $RELEASE_TAG"
  }
}
\end{lstlisting}

\section{Testing and Verification}
\subsection{Local preparation}
Figure~\ref{fig:env-file} captures the completed `.env` file populated with the Docker Hub coordinates, and Figure~\ref{fig:npm-install} shows `npm install` succeeding prior to containerization.
\begin{figure}[H]
  \centering
  \includegraphics[width=0.65\linewidth]{env-file.png}
  \caption{Environment file populated with blue/green image references.}
  \label{fig:env-file}
\end{figure}
\begin{figure}[H]
  \centering
  \includegraphics[width=0.65\linewidth]{npm-install.png}
  \caption{Dependencies installed locally before containerizing.}
  \label{fig:npm-install}
\end{figure}

\subsection{Application preview}
Figures~\ref{fig:curl-home} and \ref{fig:ui-blue} document the HTML served by the Node.js process and the styled UI indicating the blue stack metadata.
\begin{figure}[H]
  \centering
  \includegraphics[width=0.85\linewidth]{curl-home-html.png}
  \caption{`curl http://localhost:3000` returning the rendered HTML/CSS.}
  \label{fig:curl-home}
\end{figure}
\begin{figure}[H]
  \centering
  \includegraphics[width=0.6\linewidth]{ui-blue.png}
  \caption{Browser view of the blue stack showing version and host info.}
  \label{fig:ui-blue}
\end{figure}

\subsection{Container build and registry proof}
Figure~\ref{fig:build-push} shows the helper script building and pushing `yuvan4525/bluegreen-demo:blue`, while Figure~\ref{fig:dockerhub} confirms the tag in Docker Hub.
\begin{figure}[H]
  \centering
  \includegraphics[width=0.9\linewidth]{build-push-script.png}
  \caption{`./scripts/build-and-push.sh blue` output including digest.}
  \label{fig:build-push}
\end{figure}
\begin{figure}[H]
  \centering
  \includegraphics[width=0.75\linewidth]{dockerhub-tag.png}
  \caption{Docker Hub repository showing the freshly pushed tag.}
  \label{fig:dockerhub}
\end{figure}

\subsection{Blue/green stack validation}
Figure~\ref{fig:compose-ps} captures `docker compose ps` with both app containers plus the proxy, while Figure~\ref{fig:proxy-health} shows the proxy `/health` endpoint reporting the blue environment.
\begin{figure}[H]
  \centering
  \includegraphics[width=0.9\linewidth]{compose-ps.png}
  \caption{Compose stack running blue, green, and proxy services locally.}
  \label{fig:compose-ps}
\end{figure}
\begin{figure}[H]
  \centering
  \includegraphics[width=0.65\linewidth]{proxy-health.png}
  \caption{Proxy health endpoint exposing blue metadata before the switch.}
  \label{fig:proxy-health}
\end{figure}

\subsection{Green deployment and traffic switch}
Figures~\ref{fig:deploy-green}, \ref{fig:green-health}, and \ref{fig:proxy-green} document the CLI deployment to the idle color, the `/health` check on port 3002, and the subsequent switch that routes users to the green UI.
\begin{figure}[H]
  \centering
  \includegraphics[width=0.9\linewidth]{deploy-green-cli.png}
  \caption{`deploy-color.sh` rolling out the green environment.}
  \label{fig:deploy-green}
\end{figure}
\begin{figure}[H]
  \centering
  \includegraphics[width=0.65\linewidth]{green-health.png}
  \caption{Green stack `/health` response confirming version 1.0.1.}
  \label{fig:green-health}
\end{figure}
\begin{figure}[H]
  \centering
  \includegraphics[width=0.65\linewidth]{proxy-green-ui.png}
  \caption{Proxy-backed UI after switching traffic to the green environment.}
  \label{fig:proxy-green}
\end{figure}

\subsection{Jenkins pipeline evidence}
Figures~\ref{fig:jenkins-stages}--\ref{fig:jenkins-console} summarize the Jenkins success: the stage view, permalinks widget, and console log snippet covering docker login/push.
\begin{figure}[H]
  \centering
  \includegraphics[width=0.7\linewidth]{jenkins-stages.png}
  \caption{Blue Ocean stage visualization showing every stage green.}
  \label{fig:jenkins-stages}
\end{figure}
\begin{figure}[H]
  \centering
  \includegraphics[width=0.45\linewidth]{jenkins-permalinks.png}
  \caption{Job permalinks confirming build \#11 as the latest stable run.}
  \label{fig:jenkins-permalinks}
\end{figure}
\begin{figure}[H]
  \centering
  \includegraphics[width=0.85\linewidth]{jenkins-console.png}
  \caption{Console log excerpt covering the docker push executed by Jenkins.}
  \label{fig:jenkins-console}
\end{figure}

\section{Results}
The lab delivers a reusable workflow:
\begin{itemize}
  \item Deterministic Node.js container image published to Docker Hub.
  \item Docker Compose stack that keeps blue and green environments warm while an NGINX proxy steers production traffic.
  \item Jenkins pipeline that automates build, push, deploy, smoke test, and promotion with a single click.
\end{itemize}

\end{document}
